\section{Introduction}
	\setcounter{page}{1}


This report is written as a bachelor thesis by computer science students at NTNU. The project revolves around upgrading and expanding features of a multi platform app called “Stedr” which is currently in beta. Stedr's purpose is to enable people to share their stories about places around the world. This can be anything from a famous attractions to just an ordinary building Trondheim. The contributors will be able to share stories and media through external services like Digitalt Fortalt, Flickr, Instagram, Soundcloud etc. With the application, users can view other peoples stories and images  to help them explore a certain place.

\subsection{The subject: IT2901}
\emph{IT2901: prosjektarbeid i informatikk} is the name of the course this project is a part of. Through this project the students will work in groups on a specific project within the scope of informatics. The institute will recomend chosen tasks for the students to choose from. From here the students work self-reliantly under the supervision of employees at the institute. \textit{(Text based on the information gathered from the study guide.)}\\ The main purpose of the project is for the students to aquire practical experience in the software engineering process. Through working with a real customer in a team throughout the entire process the student gain valueable experience to prepare them for their future carreers after the studies.

\subsection{Stedr}
The app “Stedr” are created by a previous group as a bachelor thesis, in the same subject, on the initiative from SINTEF. The goal of the app is to make room for a comunity of contributors to share stories about public places in their area, to be displayed in the app. 

\subsection{Stakeholders}
In this subsection we will present the main people involveld with the project. 

\subsubsection{The team}
The team consists of six students all taking a Bachelor degree in Informatics at The Norwegian University of Science and Technology (NTNU). In our team we have a great variation in areas of expertise and knowledge which hpelped us greatly during the course of the project. Having expertise in many different areas we could help each other and share knowledge across the group to make everybody suited to different tasks. The importance of this project made the whole group very motivated to succeed and make for a good result.
We are:\\

\begin{tabular}{r|p{11cm}}
\textbf{Hallvard Jore Christensen} & \texttt{hallvarc@stud.ntnu.no}\\[6pt]
\textbf{Jon-André Brurberg} & \texttt{jonandbr@stud.ntnu.no}\\[6pt]
\textbf{Jørgen Rugelsjøen Wikdahl} & \texttt{jorgenrw@stud.ntnu.no}\\[6pt]
\textbf{Tor Barstad} & \texttt{torob@stud.ntnu.no}\\[6pt]
\textbf{Vegard Storm} & \texttt{vegs@stud.ntnu.no}\\[6pt]
\textbf{Øyvind Hellenes} & \texttt{oyvihell@stud.ntnu.no}\\
\end{tabular}

\subsubsection{Customer}
Our customer is SINTEF (The Foundation for Scientific and Industrial Research). They are the largest independent research organisation in Scandinavia. The organiszation was established at the Norwegian Institute of Technology (NTH) in Trondheim in 1950 and expanded rapidly in the following years.

\begin{tabular}{r|p{11cm}}
\textbf{Jaqueline Floch} & \emph{Project Manager}   \texttt{Jacqueline.Floch@sintef.no}\\[4pt]
& Our main contact inside SINTEF and Coordinator of the project from the customers side. She is also the primary driver for the app this project evolves around. \\[8pt]
\textbf{Babak Farshchian} & \emph{Interim Manager}   \texttt{Babak.Farshchian@sintef.no}\\[4pt]
& After some unfortunate events made our main contact unavailable in the very beginning, Babak took over for a period of time until Jaqueline could return as our main contact. \\
\end{tabular}

\subsubsection{Course Staff}
We also have great support from the university during the course of this project, mainly from the course staff consisting of assistants, lecturers etc.

\begin{tabular}{r|p{11cm}}
\textbf{Mohsen Anvaari} & \emph{Supervisor}   \texttt{mohsena@idi.ntnu.no}\\[6pt]
& Supervises our group during the project, giving feedback and support. \\[8pt]
\textbf{Monica Divitini} & \emph{Course co-ordinator}   \texttt{divitini@idi.ntnu.no}\\[6pt]
& Is the course co-ordinator and in charge of the subject.\\
\end{tabular}

\subsection{Report Structure}

\textbf{Chapter 1}\\
 The introduction chapter. Presenting the course, project and the different people involveld in the project.\\[20pt]
\textbf{Chapter 2}\\
Chapter describing the pre-study phase of the project. Since this project is based on an existing project with a beta product, this phase was important for us. This process is documented in this chapter summing up all our research. \\[20pt]
\textbf{Chapter 3}\\
This chapter describes the basis of our project with the main focus on the projects' structure. This includes how both the group and the project have been organized.\\[20pt]
\textbf{Chapter 4}\\
 This chapter consistes og a SRS describing the behaviour of the system to be further developed, including a detailed descripion with supplementation of diagrams and tables.\\[20pt]
\textbf{Chapter 5}\\
In this chapter we are presenting the system focusing on the architectural part.\\[20pt]
\textbf{Chapter 6}\\
This chapter focuses on the testing phase of the project with the respective results enclosed along with test descriptions.\\[20pt]
\textbf{Chapter 7}\\
 This chapter contains enclosed documents.\\[20pt]
 \textbf{Chapter 8}\\
Attachments.\\


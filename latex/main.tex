\documentclass[12pt,a4paper,titlepage]{article}
\usepackage[utf8]{inputenc}
\usepackage[T1]{fontenc}
\usepackage[english]{babel}
\usepackage{xpatch}
\usepackage{babel}
\usepackage{titlesec}
\usepackage{csquotes} 
\usepackage{filecontents}
\usepackage[style=authoryear,maxcitenames=2,backend=bibtex]{biblatex}
\usepackage{tabulary}
\usepackage{tabularx}
\usepackage{lscape}
\usepackage{rotating}
\usepackage{longtable}
\usepackage{lipsum}
\usepackage{tikz}

\usepackage{attachfile}

\usepackage{tikz}

%Section design
\definecolor{gray75}{gray}{0.75}
\newcommand{\hsp}{\hspace{20pt}}
\titleformat{\section}[hang]{\Huge\bfseries}{\thesection\hsp\textcolor{gray75}{|}\hsp}{0pt}{\Huge\bfseries}

\usetikzlibrary{calc}
\usetikzlibrary{decorations.pathmorphing}

%Header
\usepackage{fancyhdr}

\lhead{}
\rhead{\leftmark}

%\pagestyle{fancy}
%\fancyhf{}

%\renewcommand\headrule{
%\begin{minipage}{1\textwidth}
%\hrule width \hsize \kern 1mm \hrule width \hsize height 2pt 
%\end{minipage}}%

%\lhead{\thesection}
%\rhead{\section}

%\newcommand\HRule{\rule{\textwidth}{1pt}}

% Pretty much all of the ams maths packages
\usepackage{amsmath,amsthm,amssymb,amsfonts}
% Allows you to manipulate the page a bit
\usepackage[a4paper]{geometry}
% Pulls the page out a bit - makes it look better (in my opinion)
\usepackage{a4wide}
% Removes paragraph indentation (not needed most of the time now)
\usepackage{parskip}
% Allows inclusion of graphics easily and configurably
\usepackage{graphicx}
% Provides ways to make nice looking tables
\usepackage{booktabs}
% Allows you to rotate tables and figures
\usepackage{rotating}
% Allows shading of table cells
\usepackage{colortbl}
% Define a simple command to use at the start of a table row to make it have a shaded background
\newcommand{\gray}{\rowcolor[gray]{.9}}
\usepackage{textcomp}
% Provides commands to make subfigures (figures with (a), (b) and (c))
\usepackage{subfigure}
% Typesets URLs sensibly - with tt font, clickable in PDFs, and not breaking across lines
\usepackage{url}
% Makes references hyperlinks in PDF output
\usepackage{hyperref}
% Provides ways to include syntax-highlighted source code
\usepackage{listings}
\lstset{frame=single, basicstyle=\ttfamily}
% Provides good access to colours
\usepackage{color}
\usepackage{xcolor}

% Simple command I defined to allow me to mark TODO items in red
\newcommand{\todo}[1] {\textbf{\textcolor{red}{#1}}}
% Allows fancy stuff in the page header
\usepackage{fancyhdr}
\pagestyle{fancy}
% Vastly improves the standard formatting of captions
\usepackage[margin=10pt,font=small,labelfont=bf, labelsep=endash]{caption}
\newcommand{\sectionbreak}{\clearpage}

\usepackage{array}
\newcolumntype{i}[1]{>{\raggedright\let\newline\\\arraybackslash\hspace{0pt}}m{#1}}
\newcolumntype{o}[1]{>{\centering\let\newline\\\arraybackslash\hspace{0pt}}m{#1}}
\newcolumntype{j}[1]{>{\raggedleft\let\newline\\\arraybackslash\hspace{0pt}}m{#1}}


\addbibresource{references.bib}



\begin{document}

\begin{titlepage}
  \begin{center}
    \includegraphics[scale=0.5]{res/logo_ntnu_eng.png}
    
    \vspace{1cm}
    
    {\large Bachelor's Thesis}\\[1mm]
    {\large Information Science}
    
    \vspace{1cm}
    
    {\huge STEDR\\[1ex]
     SINTEF Storytelling}
    
    \vspace{1cm}
    
    {\large Hallvard Jore Christensen}\\
    {\large Jon-Andre Brurberg}\\
    {\large Jørgen Rugelsjøen Wikdahl}\\
    {\large Tor Barstad}\\
    {\large Vegard Storm}\\
    {\large Øyvind Hellenes}\\
    
    \vspace{\fill}
    
    {\large
      \ifcase\month\or
      January\or February\or March\or April\or May\or June\or
      July\or August\or September\or October\or November\or December\fi,
      \number\year
    }
    
    \vspace{\fill}
    
    Supervisor: Mohsen Anvaari
    
    \vspace{0.3cm}
    
    \vspace{\fill}
    
    NORWEGIAN UNIVERSITY OF SCIENCE AND TECHNOLOGY
    
    DEPARTMENT OF COMPUTER AND INFORMATION SCIENCE
    
    \vspace{.5cm}
    
   Sem Sælands vei 7-9\\
   
   7491 Trondheim, Norway
    
  \end{center}
\end{titlepage}



\begin{abstract}
Through this project SINTEF wanted to further develop a multi-platform application mainly for handheld devices called Stedr. The aim of the project was to add more important functionality to the application, increase usability and generally improving it in all aspects. In this report we will describe the whole process from preliminary work and planning, the process and then present the final product with testresults etc. enclosed. The application is written in Titanium Studio using mainly XML, Java and Javascript. The application makes use of many different APIs and frameworks to make use of existing services to reduce the nessecary maintanance. This was requested by the customer.
During the course of the project we experienced and overcame many challenges complicating the project, but in the end we reached most of our goals and ended up a result we are happy with. Multiple new features have been added, including support for collections and sound implementation.
\end{abstract}

\restoregeometry

%\cleardoublepage

		%Content list
		\pagestyle{empty}
		\tableofcontents
		\addtocontents{toc}{~\hfill\textbf{Page}\par}
		\chapter{}
		\cleardoublepage
		
		%Table list
%		\addcontentsline{toc}{section}{\listtablename}
		\pagestyle{empty}
		\phantomsection
		\listoftables
		\clearpage
		
		%Figure List
		\pagestyle{empty}
		\phantomsection
%		\addcontentsline{toc}{section}{\listfigurename}
		\listoffigures
		\clearpage
		
\pagestyle{fancy}			

\section{Introduction}
\thispagestyle{plain}
	\setcounter{page}{1}


This report is written as a bachelor project by computer science students at NTNU. The project revolves around upgrading and expanding features of a multi platform app called “Stedr” which is currently in beta. Stedr's purpose is to enable people to share their stories about places around the world. This can be anything from a famous attractions to just an ordinary building Trondheim. The contributors will be able to share stories and media through external services like Digitalt Fortalt, Flickr, Instagram, SoundCloud etcetera. With the application, users can view other peoples stories and images  to help them explore a certain place.

\subsection{The subject: IT2901}
\emph{IT2901: prosjektarbeid i informatikk} is the name of the course this project is a part of. Through this project the students will work in groups on a specific project within the scope of informatics. The institute will recommend chosen tasks for the students to choose from. From here the students work self-reliantly under the supervision of employees at the institute. \textit{(Text based on the information gathered from the study guide.)}\\ The main purpose of the project is for the students to acquire practical experience in the software engineering process. Through working with a real customer in a team throughout the entire process the student gain valuable experience to prepare them for their future careers after the studies.

\subsection{About Stedr}
The app “Stedr” was created in a collaboration between a group of Computer Science students from NTNU and Jacqueline Floch from SINTEF. 

Stedr is an app that connects places and stories. It combines the formal history with the social media experience. The latter will also help to create a network effect. 
We see Stedr as the first step towards a national effort for documenting narratives of places. Take the statue of Olav Tryggvason for instance. Here you can write a story about Olav, the building process, or something on the debate about removing it. That being said, Stedr is made to experience - not for creating content. To create a story takes time. It requires finding sources, put together materials, editing , etcetera. This is not something that can be performed easily on a smartphone.

Ideally we wanted to create a social network for cultural heritage - a kind of GoGoBot for cultural heritage - but this is a much more extensive project. The question is also who would drive such a platform? Kulrurådet offers a platform called Digitalt Fortalt for stories. This is unprecedented and no other countries in Europe offers something like this. But unfortunately, Digitalt Fortalt has many limitations and it is a bit old fashioned. Europeana is an alternative that has support for user generated content and might thus be used in Stedr in the future.

The main goal of Stedr is to engage people more in the cultural heritage by
\begin{enumerate}
\item Giving them easy access to stories related to a cultural heritage. \item Providing different narratives for people who have different interests like history, art, sports, music etc. \item Utilizing network effects to increase awareness about places.
\end{enumerate}

Only about half of Europe's population visited a cultural venue in 2010. This is not a satisfying statistic and hopefully in time, Stedr will help to improve this by making people aware of interesting places. A place in Stedr is a point of culture heritage. All places have several stories associated with them. That means there is a potential for providing various stories from each place.

To this date, there is many books related to locations, buildings and art in Trondheim, but several of these books are no longer available in book stores. It is also very inconvenient to walk around with books all the time. Is not it amazing that we do not have access to all this information on the mobile in 2013? Documentation and dissemination of culture in the countryside is of course challenging. There are countless places, many of which lie outside the responsibility of institutions. Associations and individual enthusiasts have helped to gather information and document the places, but the results are fragmented.

So the first step with Stedr is to create a great cultural user experience. To provide the opportunity to discover new places. The second step would be to engage people to tell about places around them.


\subsection{Stakeholders}
In this subsection we will present the main people involved with the project. 

\subsubsection{The team}
The team consists of six students all taking a Bachelor degree in Informatics at The Norwegian University of Science and Technology (NTNU). In our team we have a great variation in areas of expertise and knowledge which helped us greatly during the course of the project. Having expertise in many different areas we could help each other and share knowledge across the group to make everybody suited to different tasks. The importance of this project made the whole group very motivated to succeed and make for a good result.
We are:\\

\begin{table}[!ht]
\begin{tabular}{r|p{11cm}}
\textbf{Hallvard Jore Christensen} & \texttt{hallvarc@stud.ntnu.no}\\[6pt]
\textbf{Jon-André Brurberg} & \texttt{jonandbr@stud.ntnu.no}\\[6pt]
\textbf{Jørgen Rugelsjøen Wikdahl} & \texttt{jorgenrw@stud.ntnu.no}\\[6pt]
\textbf{Tor Barstad} & \texttt{torob@stud.ntnu.no}\\[6pt]
\textbf{Vegard Storm} & \texttt{vegs@stud.ntnu.no}\\[6pt]
\textbf{Øyvind Hellenes} & \texttt{oyvihell@stud.ntnu.no}\\
\end{tabular}
\captionsetup{textformat=empty,labelformat=blank}
\caption[The Team]{}
\end{table}

\subsubsection{Customer}
Our customer is SINTEF (The Foundation for Scientific and Industrial Research). They are the largest independent research organisation in Scandinavia. The organisation was established at the Norwegian Institute of Technology (NTH) in Trondheim in 1950 and expanded rapidly in the following years.

\begin{table}[!ht]
\begin{tabular}{r|p{11cm}}
\textbf{Jaqueline Floch} & \emph{Project Manager}   \texttt{Jacqueline.Floch@sintef.no}\\[4pt]
& Our main contact inside SINTEF and Coordinator of the project from the customers side. She is also the primary driver for the app this project evolves around. \\[8pt]
\textbf{Babak Farshchian} & \emph{Interim Project Manager}   \texttt{Babak.Farshchian@sintef.no}\\[4pt]
& After some unfortunate events made our main contact unavailable in the very beginning, Babak took over for a period of time until Jaqueline could return as our main contact. \\
\end{tabular}
\captionsetup{textformat=empty,labelformat=blank}
\caption[Customer]{}
\end{table}

\subsubsection{Course Staff}
We also have great support from the university during the course of this project, mainly from the course staff consisting of assistants, lecturers etcetera.

\begin{table}[!ht]
\begin{tabular}{r|p{11cm}}
\textbf{Mohsen Anvaari} & \emph{Supervisor}   \texttt{mohsena@idi.ntnu.no}\\[6pt]
& Supervises our group during the project, giving feedback and support. \\[8pt]
\textbf{Monica Divitini} & \emph{Course co-ordinator}   \texttt{divitini@idi.ntnu.no}\\[6pt]
& Is the course co-ordinator and in charge of the subject.\\
\end{tabular}
\captionsetup{textformat=empty,labelformat=blank}
\caption[Course Staff]{}
\end{table}

\clearpage
\subsection{Report Structure}

\textbf{Chapter 1}\\
 The introduction chapter. Presenting the course, project and the different people involved in the project.\\[20pt]
\textbf{Chapter 2}\\
Chapter describing the pre-study phase of the project. Since this project is based on an existing project with a beta product, this phase was important for us. This process is documented in this chapter summing up all our research. \\[20pt]
\textbf{Chapter 3}\\
This chapter describes the basis of our project with the main focus on the projects' structure. This includes how both the group and the project have been organized.\\[20pt]
\textbf{Chapter 4}\\
This is the Software Requirements Specification (SRS) for the new version of ``Stedr'', and may be read both as a single chapter in this report or a stand-alone document. The SRS describes the behaviour of the system consisting of a detailed description with supplementation of diagrams and tables.\\[20pt]
\textbf{Chapter 5}\\
In this chapter we are presenting the system focusing on the architectural part.\\[20pt]
\textbf{Chapter 6}\\
Here you will be presented with the whole implementation process, containing all the Sprints describing how the project developed during the project period, through the iterations.\\[20pt]
\textbf{Chapter 7}\\
This chapter focuses on the testing phase of the project. Acceptance-, case- and non-functional requirements testing are among what is on the agenda in this section.\\[20pt]
 \textbf{Chapter 8}\\
Attachments.\\[20pt]
After theses chapters the appendices follows, containing other important documents and charts.


\section{Pre-study}
	\subsection{The stedr application}
		\subsubsection{Existing functionality}
Since the application already is considered a working prototype, we will provide a list which gives a description for the functionality. Working functionality is in this report defined as the functionality that is implemented in the frontend or backend. If something is implemented backend it has to be used frontend. A more detailed techincal description is found in the architecture-section.

\begin{figure}[h!]
\begin{center}
\includegraphics[scale=0.45]{ntoverview-architecture}
\caption{A simple overview of the architecture}
\end{center}
\end{figure}

\begin{itemize}
\item Browse a map and zoom in and out.
\item Load \textbf{places}.
\item Click on a place in the map and access \textbf{stories} from Digitalt Fortalt.
\item Get social media related to a place from the content providers Instagram and Twitter.
\item Go to a users exact position on a map.
\item Search for a location in the map.
\end{itemize}
	
		\subsubsection{Limitations}
There are some limitations to the system that needs to be further developed, and some that probably would require total architectural review of the project to be fixed. Our task is to continue the development of the applicaton. An overview of the features we are going to improve are discussed in the requirements-section. Flaws that arised during the development which requires a new architecture will be discussed in the conclusions-section under Recommendations.
\section{Project organization}
	\subsection{Responsibility Areas}

Good delegation of responsibilities helps so that someone at all times have an overview over what tasks needs to be done in specific areas. This also makes it easier to estimate workloads and delegate tasks during the group meetings. It is important to note that even though there are specific responsibility areas, all the group members will be able to get practical experience in all of the project areas, even though the time spent in different areas will be distributed individually according to the responsobility areas.

\begin{description}
\item[Øyvind Hellenes - Scrummaster] Øyvind was selected as the scrum master because of his leadership qualities and because he early on took an interest in the organizatorial part of the project.
\item[Jon-André Brurberg - Project leader] Jon-André is the driving force in this team and hence, he is also the project leader. Additionally, Jon Andre showed interest in the documentation so he is also responsible for this.
\item[Tor Økland Barstad - Technical coordinator] Tor, with his compentance and knowledge of programming, is our technical coordinator and will thus oversee the code and functions as a technical supervisor for the group.
\item[Jørgen Rugelsjøen Wikdahl - Testing manager] Jørgen is responsible for testing the application to make sure the application has as few bugs as possible.
\item[Hallvard Jore Christensen - Report coordinator] Hallvard have the responsibility of managing the report. This is because of his previous expiriance with report documentation.
\item[Vegard Storm - Usability manager] Vegard showed interest in making sure the application is as user friendly as possible and will manage that aspect of the project.
\end{description}
	
	\subsection{Process model}
For the process model we chose to use the Scrum framework. This was the most natural choice for us amongst the agile methods since it’s a system we all have experience with through previous projects. There are many advantages working with Scrum. It gives clear priority for features and deadlines, which will allow us to focus more of our energy on other vital tasks. This approach promotes communication and transparency. All the team members as well as the client always knows what’s going on and the current tasks’ development through the product backlog. With the backlog cards, the whole production team is also involved with the overall time estimate, which makes it fairly accurate and controllable.

We considered a few other methods as well, like kanban and XP, but came to the conclusion that Scrum was the system for us. This was due to Scrums many structured rules which brings order, but still allows us the freedom we might need during the projects development.

With Scrum we’ll work in iterations called “Sprints” which are typically a week or two, we also stibe towards making these sprints incremental. Doing this, the model is designed, implemented and tested incrementally, feature by feature, until the project is finished. The advantage here is that for every sprint we have a working product to show for, which is a good referance to have, both for ourselves and the customer. 

Since we already have a working project from the very beginning for us to further develop, there is some obvious phase partitions. The first consists mainly on assessing the current version of the product and define the path ahead before we start the actual programming. This will be done through thorough dialog and discussion with SINTEF, to give us a unison idea of where the product are heading. User evaluation is also important in this phase, both internally and externally within the target user group. And of course technology and framework selection. After this comprehensive planning, the actual coding phase can begin. The sprints will be a big part of this, and since we are working incrementally; So we will do with the testing. Following this: the evaluating phase. In which user tests hopefully will force as many problems and bugs with the early version to surface, for us to correct.

Prototypes through a digital mock-up will be important in the planning phase. We have chosen to use Balsamiq for this, which will mean we’ll have an interactive prototype mock-up to show the customer, and should also make sure we're all on the same page. This makes it easier to have something concrete/”physical” as a reference.

\subsection{Development Environment}

Since our project is based on further developing on an existing product, there’s an advantage in using the same main framework as the previous developers. We decided to use Titanium to easily develop a multi-platform app, but we also took a close look at other options (like phonegap) and compared them meticulously in their most critical aspects. With the Titanium framework we use the Titanium SDK which is based on eclipse but tailored for it. For sharing code, Git was our system of choice, mainly because we were already familiar with it, and know it has all the functionality we could need throughout the project. Other documents and files, like notes, summaries, etc. we decided to share through a dedicated Google Drive and Dropbox folder, because each has its own advantages in different aspects. As SCRUM service we first choose Agilefant, but later decided to just use spreadsheets instead.

This project obviously involves working with a big set of APIs, like social media, dictionary and other media services. These’ll play a great part of the development and introduce other frameworks we’ll have to account for. For communication we often use mail and chat-services, but we prefer more “personal” forms of communication like a video chat through skype, phone calls and/or ideally, meetings in person.

\subsection{Work Breakdown Structure}
This Work Breakdown Structure is an overview of what we have done and how our work is distributed between packages. It is updated to match our end result.
We have given percentages to each task to show how much work have been put into it.

\begin{figure}[h!]
\begin{center}
\includegraphics[scale=0.6]{WBS}
\caption{A simple technical overview of the architecture}
\end{center}
\end{figure}

\subsection{Project Planning}
\subsection{Example of Status report and Activity plan}

%\subsubsection{Activity Plan}

Activity Plan VI: \attachfile{res/ActivityPlan_VI_Updated.pdf} %% displays "show me"
%\includepdf[pages={1}]{res/ActivityPlan_VI_Updated.pdf}

%\subsubsection{Status report}

%--------------Status report-------------

\newpage
\thispagestyle{empty}
\begin{tikzpicture}[remember picture, overlay]
	\draw[line width = 1pt] ($(current page.north west) + (0.5in,-0.5in)$) rectangle ($(current page.south east) + (-0.5in,0.5in)$);
\end{tikzpicture}

SINTEF\\
\begin{centering}
Storytelling\\
\huge{Summary Status Report}\\
\end{centering}

\vspace{1cm}
\textbf{1. Introduction}\\

The last week has been pretty busy with a lot of progress in terms of development. The last customer meeting turned out to be very constructive and the customer seemed to be satisfied with the recent development of the system.\\[6pt]

\textbf{2. Progress summary}\\

\href{https://docs.google.com/spreadsheet/ccc?key=0Aigjd3Z4fa8YdGhtR0tqN0d4Q19TR0FmQ2pTVTR1X1E&usp=drive_web#gid=0}{Updated Activity Plan (8)}

\href{https://docs.google.com/spreadsheet/ccc?key=0AlhGbQmvU9bddHl2VWVjWHhiX3VMREc0NWY4UE5hd3c&usp=sharing}{Milestone/Gantt}\\[6pt]

\textbf{3. Open / Closed problems}\\

The last week has almost been problem free, and the progress has been pretty good. Regarding the continued development there has been a delay, because of some content which haven’t been made available to the group from the customer.\\

In the comming week it will be difficult to keep up the momentum from last week, as Easter is approaching (unavailability because of travelling) and a couple of the group members are in China for a school excursion. The group members that still are available will therefore focus on few assignments which they are familiar with from earlier.\\[6pt]

\textbf{4. Planned work for next period}\\

\href{https://docs.google.com/spreadsheet/ccc?key=0AqgF_sCiXohadDQ4SlZKTTZ1ZWZ6djF2dllaZGRPSGc&usp=drive_web#gid=0}{New Activity Plan (9)}

\textbf{5. Updated risk analysis}\\

\href{https://docs.google.com/spreadsheet/ccc?key=0AlhGbQmvU9bddGs0M0tBd3ZaNUxvMnBaQnNLb0FoWnc&usp=drive_web#gid=0}{Risk analysis}

\clearpage

%-------------------------------------------

\subsection{Deviations and lessons learned}
In terms of management, we have learned some important lessons and made some adjustments to our organisatory model during the course of this project. 

With our process modell, Scrum, we found that following it by the book, became very troublesome. Therfore we decided to make some modification to the original model. The main problem with using Scrum by the letter is that we are all students, and this project only counts for half of the semesters study points. This means we all have different schedules, and thus making it difficult to have daily meetings. End meetings, or retrospective meetings is also something we haven’t prioritized much.

In Scrum it is common to use something called planning poker when deciding how long tasks should take. This essentially means that everyone “votes” on how time consuming they think a given task will be. We found this to be a little unnecessary because its usually so imprecise, and have therefore chosen to just let the persons responsible give their judgements to save time. 

As for lessons, we have especially learned about the importance of clear milestones. In the beginning we had very unclear goals and this affected the group. It was later solved when we got a priority list from our cutomer. Working without a clear focus can be challenging for the team members.

We have also learned how crucial it is to have good communication. Because of a misunderstanding between two of our customers we spent a week working on the user interface of Stedr when we should have prioritised integrating API’s instead.
\section{System Requirements Specification}
\thispagestyle{plain}

\subsection{Purpose}
This is the software requirement specification for the new version of Stedr, both the backend system that provides content and also the frontend that shows the content and the context of the content to the user. Here the traditional architectural terms backend - and frontend are used, but there are some subtleties to this term, as the frontend itself is managing a content service of its own. 

\subsection{Intended audience and reading suggestions}
Intended readers for this document are current and future developers, and the customer. The reader should also be noted that the SRS both can be read as a stand-alone document to get an overview of the rationalization behind the development process, but that it also is a part of the project report as a whole


\subsection{References}
The software requirements are based on the standard as provided by ISO/IEC:25010 \cite[10]{25010}, and also the models that can be found in this report’s section for architecture and modelling. References to the ISO-standard and other literature are found at the end of the project report under references.

\subsection{Product perspective}
Originally Stedr is a product developed by students at NTNU as a part of the subject TDT4290, and this application will form a basis for out continued development. The state of the exisiting application is considered to be a working prototype, and to some degrees it is an application that is built up with a traditonal server-client architecture. A simple technical overview of the system is provided below. 

\begin{figure}[h!]
\begin{center}
\includegraphics[scale=0.6]{tooverview-architecture}
\caption{A simple technical overview of the architecture}
\end{center}
\end{figure}

\subsection{User classes and charateristics}
The users of the program mainly divide into two categories. One of those groups is the primary user group which are interacting with the smartphone application, frontend, to see content. A typical primary user is an highschool student which is introduced to the program in the context of cultural heritage awareness. As mentioned earlier, the broad goal of the application is to make people aware about cultural heritage. If that goal is fulfilled, some primary users of the application will hopefully transit over to become a content provider. 

\noindent

A content provider has the possibility to interact with the system directly, but also he or she can choose to interact with the system more indirectly. This varying degree of interaction will hopefully lower the threshold for users transisting from content consumer to content provider, which is the overall goal of the application. 

\noindent

Another secondary user is the maintainter-administrator. The maintainer-administrator will use a special set of tools to approve creation of the systems places, but these tools are provided by the external system Flickr. Our internal system is communicating with that external system so that applications relevant information is sent from the external to the internal system, but in the end the external system is stand-alone and can not be controlled directly from the internal system. 

\noindent

\subsection{Product functions}

The main features of the program for different user categories are presented as a bullet list below. All of the provided functions by the system are available to every user without the need of registration or approval by the systems maintainer-administrator with one exception relating to adding new places. In addition the user may need to registrate on external systems to make use of those features.

\noindent


\begin{description}
\item [\textbf{Primary user}]
\end{description}
\begin{itemize}
\item See places on a world map
\item Navigate the map
\item Select places on the map, and look at stories to the related place. 
\item Select places on the map, and look at pictures to the related place.
\item Select places on the map, and find sounds to the related place.
\item Select places related to pre-defined themes (i.e: art in Trondheim).
\item Make posts to a social medium. 
\end{itemize}
\vspace{0.5cm}
\begin{description}
\item [\textbf{Secondary user, content provider}]
\end{description}
\begin{itemize}
\item Create stories and pinpoint them so that they appear in relation to a place.
\item Create sound and pinpoint them so that they appear in relation to a place.
\item Create pictures and pinpoint them so that they appear in relation to a place
\end{itemize}
\vspace{0.5cm}
\begin{description}
\item [\textbf{Secondary user, maintainer-administrator}]
\end{description}
\begin{itemize}
\item Approve content providers so they can create places.
\end{itemize}

\subsection{Operating environment}
The frontend application of the system is a smartphone application which aims to run on the two major smartphone platforms Android (2.2 and above) and iOS. Because of difficulties with developing towards the iOS platform without equipment from Apple, the goal is to get the application to run on a unspecific versions of iOS to see that a full implementation of the application on iOS is feasible. The backend of the system, or server, should run as a cloud-based platform provided by Heroku, as the case was for the existing verison of the server. Since that service now is unavailable the new version of the backend will run as a new service instead of replacing the former one. 

\subsection{Design and Implementation Constraints}
Because of the nature of the project as a part of a course, there will be few constraints regarding the development of the system. Because there already exists functionality it's natural to constraint the system to make use of the existing code and technologies. Reimplementing them with new code or technologies are allowed, but since time is a limitied resource (approximately 20 hours per member per week) it is important to be time effective. That time effectivity and available workload is also to be seen as a design and implementation constraint. Apart from a private GitHub account, the project is to be done without funding so for the deploying a free service has to be chosen.

\noindent

Since the customer is a proffesional organisation, it is also important that the system behaves correctly according to licensing. The system itself is to be open source under the Berkley Software Distribution license version 3 (BSD-3). It is also important that the system handles licensing from external systems correctly, and only shares legal content.

\subsection{User Documentation}

Documentation to system users will be provided in the application itself, this documentation has to easily be editable by the maintainer-administrator which will. In additon to the user documentation there will be provided documentation for developers as an appendix in this report, and code documentation in the code itself and in the GitHub-repository. 

\subsection{Assumptions and Dependencies}

An important assumpton in the development of the system is that the former system delivers the functionality which is stated in the feature list given by the customer. A copy of this list can be found in the appendix.

\noindent

Another important assumption, is that the external systems that were implemented in the earlier systems still is functional. This is also a dependency, because changes in those external systems will make internal system malfunctioned. This can be seen as a large drawback in the system, but as the backend is to be kept to a minimal external sources have to be used for content storage and content providing.

\subsection{System Features}

\clearpage

\begin{table}[!ht]
\begin{center}
\begin{tabular}{i{4cm}|| i{10cm}} \toprule

\multicolumn{2}{c}{\textbf{SF-1}} \\ \hline

Name & Find place on map \\ \hline
Priority & H \\ \hline
Goal & To browse the map to find a given place \\ \hline
Actors & Primary User \\ \hline
Preconditions & \begin{enumerate}  \item The home screen is displayes  \item The internal system and external systems are running \item The device has a internet connection  \end{enumerate} \\ \hline
Stimulus-Response  &  \begin{enumerate}  \item The home screen is displayes  \item The internal system and external systems are running \item The device has a internet connection  \end{enumerate} \\ \hline
Alternate Flow & \begin{itemize} \item[2a] The place does not exist and is not shown on the map \end{itemize} \\ \hline
Functional Requirement & A user should be able to access and browse a map, with places as pinpoints at their respective geographical location. The pinpoints should contain the picture and information found on Flickr. Group places close to eachother in one icon on map. \\ \hline
Related Use Cases & 1,3 \\ \hline
Dependencies & none \\ \bottomrule

\end{tabular}
\end{center}
\end{table}

\begin{center}
\begin{tabular}{i{4cm}|| i{10cm}} \toprule

\multicolumn{2}{c}{\textbf{SF-2}} \\ \hline

Name & Open menu \\ \hline
Priority & H \\ \hline
Goal & Open the drawer menu  \\ \hline
Actors & Primary User \\ \hline
Preconditions & \begin{enumerate} \item 2,3 \item[4] A screen with the menu button \end{enumerate} \\ \hline
Stimulus-Response & \begin{enumerate} \item The user clicks the menu button \item The menu opens \end{enumerate} \\ \hline
Alternate Flow & \begin{itemize} \item[1a] The user clicks the menu button, and the menu is already open \item[2a] The menu closes \end{itemize} \\ \hline
Functional Requirement & A button with the possibility to open the menu should always be presented to the user, so that the user easily can navigate the application. \\ \hline
Related Use Cases & 1,2 \\ \hline
Dependencies & none \\ \bottomrule

\end{tabular}
\end{center}

\begin{center}
\begin{tabular}{i{4cm}|| i{10cm}} \toprule

\multicolumn{2}{c}{\textbf{SF-3}} \\ \hline

Name & Search for a location \\ \hline
Priority & M \\ \hline
Goal & Go to a location on the map \\ \hline
Actors & Primary User \\ \hline
Preconditions & \begin{enumerate} \item 1,2,3 \end{enumerate} \\ \hline
Stimulus-Response & \begin{enumerate} \item The user searches for a location with the search bar in the map view. \item The map navigates to the location \end{enumerate} \\ \hline
Alternate Flow & \begin{itemize} \item[2a] Location is not found and is not navigated to. \end{itemize} \\ \hline
Functional Requirement & A search bar related to the map should be presented to the user, so the user can search for locations (independent of places) to see if there are any stories at that place. \\ \hline
Related Use Cases & 1 \\ \hline
Dependencies & none \\ \bottomrule

\end{tabular}
\end{center}

\begin{center}
\begin{tabular}{i{4cm}|| i{10cm}} \toprule

\multicolumn{2}{c}{\textbf{SF-4}} \\ \hline

Name & Refresh map \\ \hline
Priority & H \\ \hline
Goal & Update the map with content. \\ \hline
Actors & Primary User \\ \hline
Preconditions & \begin{enumerate} \item 1,2,3 \end{enumerate} \\ \hline
Stimulus-Response & \begin{enumerate} \item The user clicks the update button. \item The map refreshes and show new places \end{enumerate} \\ \hline
Alternate Flow & \begin{itemize} \item[2a] No new places are found, so no places are added to the map. \end{itemize} \\ \hline
Functional Requirement & The user should be presented with a button that makes requests for new places with content when pushed. This function should also be done automatically so that new content is sent to the user within 5 minutes after it’s added. \\ \hline
Related Use Cases & 1 \\ \hline
Dependencies & none \\ \bottomrule

\end{tabular}
\end{center}

\begin{center}
\begin{tabular}{i{4cm} ||  i{10cm}} \toprule

\multicolumn{2}{c}{\textbf{SF-5}} \\ \hline

Name & Go to location \\ \hline
Priority & H \\ \hline
Goal & Go to users location. \\ \hline
Actors & Primary User \\ \hline
Preconditions & \begin{enumerate} \item 1,2,3 \end{enumerate} \\ \hline
Stimulus-Response & \begin{enumerate} \item The user clicks the gps button. \item The map zooms to the users location.  \end{enumerate} \\ \hline
Alternate Flow & \begin{itemize} \item[2a] GPS not available so it can’t go to the users location. \end{itemize} \\ \hline
Functional Requirement & Since the user has the possibility to navigate the map freely, it should also be possible to quickly navigate to places relevant (in context of location) to him/her. \\ \hline
Related Use Cases & 1 \\ \hline
Dependencies & none \\ \bottomrule

\end{tabular}
\end{center}

\begin{center}
\begin{tabular}{i{4cm}|| i{10cm}} \toprule

\multicolumn{2}{c}{\textbf{SF-6}} \\ \hline

Name & Open views \\ \hline
Priority & H \\ \hline
Goal & Open views and see the content related to that specific view \\ \hline
Actors & Primary User \\ \hline
Preconditions & \begin{enumerate} \item 1,2,3 \end{enumerate} \\ \hline
Stimulus-Response & \begin{enumerate} \item The user clicks on a view \item The user changes views at will \item Content  \end{enumerate} \\ \hline
Alternate Flow & \begin{itemize} \item[1a] If the user clicks a button for the already chosen view, nothing should happen. \end{itemize} \\ \hline
Functional Requirement & For navigation in the place view, the user should be presented with different buttons (or tabs) so that the user easily can navigate between content and still have an overview of what types of content the application provides. Preview picture gallery when places are grouped together. Add description about place, own vire for sound. Be able to show place location on map from story. Be able to filter stories by tag, author, institution video/no video. preview stories by sound from SoundCloud \\ \hline
Related Use Cases & 3 \\ \hline
Dependencies & none \\ \bottomrule

\end{tabular}
\end{center}

\begin{center}
\begin{tabular}{i{4cm}|| i{10cm}} \toprule

\multicolumn{2}{c}{\textbf{SF-7}} \\ \hline

Name & Load content \\ \hline
Priority & H \\ \hline
Goal & Content is loaded from the external systems \\ \hline
Actors & Internal System \\ \hline
Preconditions & \begin{enumerate} \item 1,2,3 \end{enumerate} \\ \hline
Stimulus-Response & \begin{enumerate} \item Access the server as done in the previous version of the system \item Provide input to the server “placeId=” \item Content is loaded and a JSON-object is replied by the server \end{enumerate} \\ \hline
Alternate Flow & \begin{itemize} \item[1a] If the user clicks a button for the already chosen view, nothing should happen. \end{itemize} \\ \hline
Functional Requirement & The API for DF has to be changed, without changing the behaviour of the response from the server. In additon to this the server will respond with a new container for the audio content. Other content should be handled as normal. Retrieve collectionfrom DF based on hashtag and location.  Retrieve stories in a collection from DF based on tags. Open info retrived from SoundCloud based on hashtags or location. Retrieve information from Instagram based on Hashtags. Be able to get tinyUrls to different content. \\ \hline
Related Use Cases & Null \\ \hline
Dependencies & none \\ \bottomrule

\end{tabular}
\end{center}

\begin{center}
\begin{tabular}{i{4cm}|| i{10cm}} \toprule

\multicolumn{2}{c}{\textbf{SF-8}} \\ \hline

Name & Collection \\ \hline
Priority & H \\ \hline
Goal & Get all places related to a theme. \\ \hline
Actors & Primary User \\ \hline
Preconditions & \begin{enumerate} \item 1,2,3 \end{enumerate} \\ \hline
Stimulus-Response & \begin{enumerate} \item Access the menu bar. \item Click on the Collections-button \item Choose a collection \item Collections view is opened \item Change to map view \item Places related to the collections is shown on map \end{enumerate} \\ \hline
Alternate Flow & \begin{itemize} \item[3a] No Collections are available \end{itemize} \\ \hline
Functional Requirement & A container called Collections are to be implemented. Collections. Allow switching between map-related and collection related funtionallity. Display picture, title and description about a collection. Have a storyListView. Preview stories in collection story list. Open story in collection list. Places on map view with icon for each story in colelction. Preview a place for story on map.\\ \hline
Related Use Cases & 3 \\ \hline
Dependencies & none \\ \bottomrule

\end{tabular}
\end{center}

\begin{center}
\begin{tabular}{i{4cm}|| i{10cm}} \toprule

\multicolumn{2}{c}{\textbf{SF-9}} \\ \hline

Name & Upload content \\ \hline
Priority & M \\ \hline
Goal & Upload content \\ \hline
Actors & Primary User \\ \hline
Preconditions & \begin{enumerate} \item 1,2,3 \item[5] The user has an account at the content provider he or she is trying to upload to.  \item[6] Places related to the collections is shown on map \end{enumerate} \\ \hline
Stimulus-Response & \begin{enumerate} \item Access the tabs for different views \item Click the add-button in the views. \end{enumerate} \\ \hline
Alternate Flow & \begin{itemize} \item[3a] No Collections are available \end{itemize} \\ \hline
Functional Requirement & The user should have the possibility to add content so that. Add picture directly from stedr. ask the user for login-credentials the first time, then store locally for continued access. A similar approach for SoundCloud. Have relevant hashtags copied to clipboard. Be able to comment and like pictures on instagram. \\ \hline
Related Use Cases & Null \\ \hline
Dependencies & none \\ \bottomrule

\end{tabular}
\end{center}

\begin{center}
\begin{tabular}{i{4cm}|| i{10cm}} \toprule

\multicolumn{2}{c}{\textbf{SF-10}} \\ \hline

Name & Get help and info \\ \hline
Priority & H \\ \hline
Goal & Be informed \\ \hline
Actors & Primary User \\ \hline
Preconditions & \begin{enumerate} \item 1,2,3  \end{enumerate} \\ \hline
Stimulus-Response & \begin{enumerate} \item Access the drawer menu \item Click the help button. \item Select the option for what help you need \end{enumerate} \\ \hline
Alternate Flow &  \\ \hline
Functional Requirement & Introduction for first users. Help available at any time.\\ \hline
Related Use Cases & Null \\ \hline
Dependencies & none \\ \bottomrule

\end{tabular}
\end{center}




\subsection{Product Quality}

Guided by ISO:25010, meetings with our supervisor and the feature list given to us by the customer the product qualities that are important for the project is functional suitability, portabilty and maintainability.

\subsubsection{Compatibility}


\subsubsection{Performance Efficiency}
Even though the system isn't a part of a critical operation, the new and improved system will have performance efficiency as an important model of quality. The reasoning behind this is that decreased response time between components in the system is specfically asked for at multiple places in the feature list provided by the customer. 

As of now the time to load new content from the content providers to the application is slow and random. Because of this there are no exact estimation on the time used to pull content from Digitalt Fortalt and Instagram, but the application should use no more than \textit{300 seconds} to pull new content. Unrelated to the goal of performance issue; the user should be informed that the application isn't a real-time application.

Requirements related to resources utilized by the application when performing it's tasks, are already met by the prototype. The new version of the application are bound also bound by these goals. Specifically the backend is bound by the resources provided by the 1x Heroku Cloud Platform. Because of the utilization of the Google Maps API, the resources frontend is limited to the bound given to the application from Google Maps. 

Regarding capacity used by the the application, there should be an improvement. Because the application is to be used on the go where there may not be any WiFi-hotspots, the application should restrain itself to dowload content that is unrelated to where the user is. Because of the varied content types, it is hard to set a defined limit in how much contents (in terms of megabytes) the application should download. The limitations given to the application will therefore be set by the equation: \\ 
\begin{center} 
$\textrm{Bound}=\textrm{Content from Digitalt Fortalt}+5\times\textrm{Content from Twitter}+10\times\textrm{Picture from Flickr}+5\times\textrm{Picture from Instagram}$
\end{center}

\subsubsection{Reliability}
Since the application is going to be online without a team responsible for the technical maintenance, the server should be operative as long as the external content providers are feeding it with content. 

Because of the early versioning of the application, the aspect of maturity is not important for this application. Users of the application are few, and they know what the capabilities of the application is. This means that a user follows a rigid pattern and within that pattern, the probability to execute faults is almost non-existing. Functionality outside that pattern is not supported and thereby it's impossible to execute mistakes.

An important charecteristic of the application is that it has to be available just as often as a professional service. This means that under normal circumstances, the uptime of the backend and front should be 99 \%

Whenever faults are occuring, it is crucial that the backend has implemented services so that it can recover without the need of a maintainer. Because of the relative simplicity of the backend, the server should restart itself within \textit{180 seconds}

\subsubsection{Portability}

It is important to the customer that the application is made available on multiple platform as this is a demand by Tag Cloud. The minimal number of platforms which the product should run on is iOS and Android. 

Following this, the frontend of the application should be written once and compiled down to both the iOS and Android platform. The backend should provide agnostic responses, so that the responses can be handled the same by on Android and iOS devices. 

Because of the early development phase of the application, there is not a requirement to install the application from the normal application providers Google Play and Apple Store. It is enough that it is possible to install the applications on development devices. This also leads to that the application doesn't need to consider replaceability at this point.  




\clearpage 					%This command flushes out all the queued floats waiting to be placed to avoid them being placed inapropriately in another section.
\section{Architecture}

The current architecture of the application is as shown in the figure 1 from the requirements-section. We have a backend written in Java that retrieves information from services like Digitalt Fortalt, Flickr and Instagram. Digitalt Fortalt is where all the stories are obtained from, Flicker holds all the locations, and the pictures are taken from Instagram based on tags. The information is stored on the server and can now be used by the client, which holds the frontend of the application that is being developed on Appcelerator Titanium, using mainly JavaScript and XML. Twitter is integrated directly into the frontend and does not have to go through the server. This is what we eventually would like to do for all the external services, and completely get rid of the backend, but given the time available for the project and the features the customer wants us to implement, this is not a task that will be developed. We would also like the user to be able to publish to more of the external services via the application. Publish a picture to instagram, add a new location to flickr, or share a story on facebook are all features we would like to add, but are not top priority given our time restrictions.

\subsection{Backend}
The Backend is written in Java and mainly retrieves data from external APIs and save it on the server so that it can be used by the application.

\begin{figure}[!h]
\begin{center}
\includegraphics[scale=0.1]{class-backend.jpg}
\caption{Backend (Look at end of report for full scale) }
\end{center}
\end{figure}

\subsection{Frontend}
The Frontend of the application is an interface to let the user enter, manipulate and view data. It’s the part of the application that is being interpreted on the users own device, and is based on XML, TSS and JavaScript for design and functionality. 

Every window in the application has a javascript-, tss- and xml-file associated with it. A window can contain various views that can each have different event listeners. What the user sees depends on the window currently open and its associated xml, tss and javascript files and what happens when interacting with a view depends on the event-listeners attached to that particular view. Interactions can be purely visual or it can trigger core functionalities. For example the refresh button on the map window has an event-listener attached to it so that when the user clicks it, it will attempt to fetch the locations from the server and plot them on the map. It will also animate the refresh icon to spin, giving the user feedback that the click was registered.

\begin{figure}[!h]
\begin{center}
\includegraphics[scale=0.1]{class-frontend.png}
\caption{Frontend (Look at end of report for full scale)}
\end{center}
\end{figure}

\subsection{Use Case}

\begin{figure}[!h]
\begin{center}
\includegraphics[scale=0.6]{ms.png}
\caption{Map View (Home)}
\end{center}
\end{figure}

\begin{figure}[!h]
\begin{center}
\includegraphics[scale=0.6]{mens.png}
\caption{Menu View}
\end{center}
\end{figure}

\begin{figure}[!h]
\begin{center}
\includegraphics[scale=0.6]{ps2.png}
\caption{Place Screen}
\end{center}
\end{figure}

\clearpage

\subsection{Sequence}

\begin{figure}[!h]
\begin{center}
\includegraphics[scale=1]{Get-Stories}
\caption{Get Stories}
\end{center}
\end{figure}

\begin{figure}[!h]
\begin{center}
\includegraphics[scale=1]{Get-Collections}
\caption{Get Collections}
\end{center}
\end{figure}
\clearpage 					%This command flushes out all the queued floats waiting to be placed to avoid them being placed inapropriately in another section.

\section{Testing}
\thispagestyle{plain}
System testing or software testing, falls into something that is called “Black-box testing”. This is a method of software testing, that investigates the functionality of the application. Eg. what it does, it is simply described as this:
It will not require to know how to code, or need any sufficient level of skill to programming when an system test is about to go down. It will neither interfere with it’s internal structure or workings. 

\subsection{Testing Procedure}

When you are about to conduct a test, you find a test-person. Then you tell them what the software is supposed to do. And give them the Test cases[7.2], and explain to them it is very important to think loud so we get the most out of the testing. 

\subsection{Test Cases}

Test cases are built around the specifications and requirements of the application. What the application is supposed to do.  

\begin{table}[htp]
\begin{center}
\begin{tabular}{ i{4cm} ||  i{10cm}} \toprule
\multicolumn{2}{c}{\textbf{Get Places}} \\ \hline
ID & T-F1 \\ \hline
Requirements & SF-1 \\ \hline
Feature& Places are shown on the map \\ \hline
Preconditions& \begin{enumerate} \item Flickr is up \item The Flickr group contains photos with locations \item Application is installed on device \item Device is connected to the internet \end{enumerate} \\ \hline
Test Description& \begin{enumerate} \item Open the application \item Wait for 30 seconds \item Click on a pinpoint \item Zoom out to a world view  \end{enumerate} \\ \hline
Expected result & The map should show some clickable pinpoints. When clicked the pinpoints should open a little box containing a thumbnail picture and small text provided by the Flickr Stedr group. \newline
When zoomed out new places should be loaded according to what the user see. \\ \hline
Pass/Fail criteria & The test is considered a pass if the expected result happens. The last step that need to be passed is that the place at Grenada is shown. \newline
If there are any incosistencies with the expected result, the test should be considered a fail. \\ \hline
Severity & High\\ \bottomrule
\end{tabular}
\end{center}
\caption{Test Case: Get Places}
\label{tab:Test Case: Get Places}
\end{table}


\begin{table}[htp]
\begin{center}
\begin{tabular}{ i{4cm} ||  i{10cm}} \toprule
\multicolumn{2}{c}{\textbf{Open menu}} \\ \hline
ID & T-F2 \\ \hline
Requirements & SF-2 \\ \hline
Feature& Drawer menu with options is opened. \\ \hline
Preconditions& \begin{enumerate} \item[ ]1,2,3,4 \end{enumerate} \\ \hline
Test Description& \begin{enumerate} \item Click on the menu button \item Click on all of the icons in the menu \item Click on the menu again \end{enumerate} \\ \hline
Expected result & When the menu button is pressed, a drawer menu should open. All of the icons in the drawer menu is also buttons and when clicked again, the menu button should close the drawer menu. \\ \hline
Pass/Fail criteria & The test is considered a pass if the menu button opens and closes a drawer menu. Also, all of the icons should \\ \hline
Severity & High\\ \bottomrule
\end{tabular}
\end{center}
\caption{Test Case: Open Menu}
\label{tab:Test Case: Open Menu}
\end{table}


\begin{table}[htp]
\begin{center}
\begin{tabular}{ i{4cm} ||  i{10cm}} \toprule
\multicolumn{2}{c}{\textbf{Views}} \\ \hline
ID & T-F3 \\ \hline
Requirements & SF-6 \\ \hline
Feature& All the views are accessible \\ \hline
Preconditions& \begin{enumerate} \item[ ]1,2,3,4 \item[T-F1] Get places \end{enumerate} \\ \hline
Test Description& \begin{enumerate} \item Click on a pinpoint \item Click on the small window that appears \item Click on one of the buttons \textit{Images, Sound, Story} \item Dependent on the prevoius step, click on the buttons not yet pushed \item Click the menu button \item Click home \end{enumerate} \\ \hline
Expected result & Which view that is selected is shown to the user by being in a different color than the two other buttons. If the button for the selected view is touched, nothing should happen. \newline
 For every button representing a non-selected view, the user should be taken to the view as indicated by the button text. \\ \hline
Pass/Fail criteria &The test is passed if the button: \newline[5pt]
Image - Takes you to the image view\newline
Story - Takes you to the story view\newline
Sound - Takes you to the sound view.\newline[5pt]
The selected view has a unclickable button in a different color representing the selected view.\newline
Considered a fail if there are any inconsistencies with the criterias above. \\ \hline
Severity & High\\ \bottomrule
\end{tabular}
\end{center}
\caption{Test Case:  Views}
\label{tab:Test Case: Views}
\end{table}


\begin{table}[htp]
\begin{center}
\begin{tabular}{ i{4cm} ||  i{10cm}} \toprule
\multicolumn{2}{c}{\textbf{Load Content}} \\ \hline
ID & T-F4 \\ \hline
Requirements & SF-7 \\ \hline
Feature& Content is loaded for the places \\ \hline
Preconditions& \begin{enumerate} \item[T-F3] Views \end{enumerate} \\ \hline
Test Description& \begin{enumerate} \item Click on a pinpoint(not Camera Obscura) \item Click on the description \item Go through the views as in T-F3 \item[3a] Click on all of the titles on the story \item[3b] Click on two random images \item[3c] Click on a sound \end{enumerate} \\ \hline
Expected result &The places should be loaded with relevant and accessible content from all of the content providers.. \newline
If some content-types aren’t provided for the specific place, the content type should be loaded but indicate that it is empty. \\ \hline
Pass/Fail criteria &The test is considered a pass if the expected result happens. \newline
If there are any inconsistencies with the expected result, the test should be considered a fail. \\ \hline
Severity & High\\ \bottomrule
\end{tabular}
\end{center}
\caption{Test Case: Load Content}
\label{tab:Test Case: Load Content}
\end{table}



\begin{table}[htp]
\begin{center}
\begin{tabular}{ i{4cm} ||  i{10cm}} \toprule
\multicolumn{2}{c}{\textbf{Collection view}} \\ \hline
ID & T-F5 \\ \hline
Requirements & SF-8 \\ \hline
Feature& Show a view with the stories related to a collection \\ \hline
Preconditions& \begin{enumerate} \item[T-F2] Open menu \item[T-F4] Load Content \item[5] It exist a collection \end{enumerate} \\ \hline
Test Description& \begin{enumerate} \item Press the menu button \item Press the Collection button \item Press a collection \end{enumerate} \\ \hline
Expected result & When the collection button is pressed a new view should open with the list of stories related to the collection. \\ \hline
Pass/Fail criteria & The test is considered a pass if it is possible to open the menu and access a collection with a list of stories. \\ \hline
Severity & Medium\\ \bottomrule
\end{tabular}
\end{center}
\caption{Test Case: Collection View}
\label{tab:Test Case: Collection View}
\end{table}


\begin{table}[htp]
\begin{center}
\begin{tabular}{ i{4cm} ||  i{10cm}} \toprule
\multicolumn{2}{c}{\textbf{Collection map view}} \\ \hline
ID & T-F6 \\ \hline
Requirements & SF-8 \\ \hline
Feature& Show places related to a collection as pinpoints in a map \\ \hline
Preconditions& \begin{enumerate} \item[T-F2] Collection View \end{enumerate} \\ \hline
Test Description& \begin{enumerate} \item Press the menu button \item Press the Collection button \item Press a collection \item Press the \textit{show on map}-button \end{enumerate} \\ \hline
Expected result &When the “show on map”-button is clicked, a map view should open with related places showed as pinpoints. Pinpoints not related to the collection should not be placed on the map. \\ \hline
Pass/Fail criteria & The test is considered a pass if all places related to a collection is exclusively shown in a map view. \\ \hline
Severity & Medium\\ \bottomrule
\end{tabular}
\end{center}
\caption{Test Case: Collect Map View}
\label{tab:Test Case: Collect Map View}
\end{table}


\begin{table}[htp]
\begin{center}
\begin{tabular}{ i{4cm} ||  i{10cm}} \toprule
\multicolumn{2}{c}{\textbf{Gallery}} \\ \hline
ID & T-F7 \\ \hline
Requirements &  \\ \hline
Feature& Gallery function\\ \hline
Preconditions& \begin{enumerate} \item[T-F4] Load Content \item[ ] The application is in aplcae with a story where there are multiple images to the story. \end{enumerate} \\ \hline
Test Description& \begin{enumerate} \item Press the story title \item If there are more pictures related to a story, press the arrows \end{enumerate} \\ \hline
Expected result &When accessing stories with multiple pictures as content, arrows indicating the possibility to go through picture files should appear. When pressed new images should replace the old picture. \\ \hline
Pass/Fail criteria &The test is considered a pass if the expected result happens. \newline
If there are any inconsistencies with the expected result, the test should be considered a fail. \\ \hline
Severity & Low\\ \bottomrule
\end{tabular}
\end{center}
\caption{Test Case: Gallery}
\label{tab:Test Case: Gallery}
\end{table}


\begin{table}[htp]
\begin{center}
\begin{tabular}{ i{4cm} ||  i{10cm}} \toprule
\multicolumn{2}{c}{\textbf{Upload Content}} \\ \hline
ID & T-F8 \\ \hline
Requirements &  SF-9\\ \hline
Feature& Content can be uploaded to Instagram, Twitter and SoundCloud \\ \hline
Preconditions& \begin{enumerate} \item[T-F4] Load Content \item[6] Successfully connected to the content (not story provider) providers \end{enumerate} \\ \hline
Test Description& \begin{enumerate} \item Click on a pinpoint \item Click on the description \item Go through the views as in T-F3 \item Upload textual content to Twitter \item Upload picture to Instagram \item Upload sound to SoundCloud \end{enumerate} \\ \hline
Expected result &The places should be loaded with relevant and accessible content from all of the content providers..\newline
If some content-types aren’t provided for the specific place, the content type should be loaded but indicate that it is empty. \\ \hline
Pass/Fail criteria & The test is considered a pass if the expected result happens. \newline
If there are any inconsistencies with the expected result, the test should be considered a fail. \\ \hline
Severity & Medium\\ \bottomrule
\end{tabular}
\end{center}
\caption{Test Case: Upload Content}
\label{tab:Test Case: Upload Content}
\end{table}

\clearpage

\subsection{Test Execution}
\subsubsection{Acceptance Testing}
Acceptance testing is one of the last levels of the software testing process. The purpose of such testing is to evaluate the system's compliance with the given requirements to check wether is is acceptable for delivery. Hence the name accepance testing.

On may 14. during one of our last meetings we sat down with Jaqueline Floch for the final acceptace test of our system. Since the project has been more based on furter developing features than improving the graphical user interface and increasing usability, we had a more feature based acceptance testing where we went through the requirement list. Not to say that usability is not as important, but in addition to have had less focus than the new features, our customer have recieved the latest build almost every week in the later stages of the project. Through this, the customer have conducted something similar to a usability based acceptance test regurlary throughout the project, to which we recieved feedback \todo{(Examples of feedback can be found in Appendix X)}. Based on this we found usability based user acceptance test, which we know is the way many of the other groups have chosen to handle the test, to be redundant.
The test was conducted by going through the requirement list sorted by priority and individually recieving a verdict on acceptance. The individual requirement either revieved \texttt{Pass, fail, outdated} or \texttt{cancel} if the requirement was pulled. The latter was the case on a lot of the requirements with lower priotity since we had little time and the customer wanted us to focus on the more important tasks.

The result of the acceptance test are listed below:\\

\includepdf[%
 	pages={-},
	offset=0in 0in,%
	addtolist={%
		\thepage, table, {Acceptance Testing}, tab:AcceptanceTest
	},
	]{res/AcceptanceTesting_Requirements.pdf}

\subsubsection{NFR testing}
It is important for the project that our result meets the projects main non-functional requirements, described in the ``Product quality'' section of the SRS chapter \todo{chapter 4.12}, for it to be considered a success. 

\paragraph{Compability}

\paragraph{Performance Efficiency}
We have greatly improved the core of the system to boost the efficiency of the application, this should cause the application to use no more than 300 seconds to pull new content from the APIs. The efficiency was tested by adding new content and recording 10 times with different content posted at different times. By measuring the individual response times and calculating the average result we will get a rough estimate.\\

\begin{table}[!htp]
\begin{center}
	\begin{tabular}{ | l | l | l | r | }
	\hline
	 \#	 	& Content 		& Time a day 		& Result \\ \hline
	 1		&Instagram		& 10:55			& \texttt{1 sec} \\ \hline
	 2		&Instagram		& 14:40			& \texttt{1 sec} \\ \hline
	 3		&Instagram		& 14:45			& \texttt{1 sec} \\ \hline
	 4		&Tweet		& 14:16			& \texttt{19 sec} \\ \hline
	 5		&Tweet		& 14:18			& \texttt{20 sec} \\ \hline
	 6		&Tweet		& 10:40			& \texttt{20 sec} \\ \hline
	 7		&Story		& 15:26			& \texttt{133 sec}\\ \hline
	 8		&Story		& 16:15			& \texttt{10 sec}\\ \hline
	 9		&Story 		& 11:00			& \texttt{104 sec}\\ \hline
	 10		&Story		& 15:48			& \texttt{135 sec}\\ \hline
	 11		&Story		& 16:03			& \texttt{107 sec}\\ \hline
	 12		&Story		& 19:01			& \texttt{132 sec}\\ \hline
	 12		&Soundcloud		& 18:51			& \texttt{5 sec}\\ \hline
	 12		&Soundcloud		& 19:15			& \texttt{7 sec}\\ \hline
	 12		&Soundcloud		& 19:20			& \texttt{6 sec}\\  
	 \hline
 	 \end{tabular}
\end{center}
\caption{Performance Efficiency: Publishing New Content}
\label{tab:Performance Efficiency: Publishing New Content}
\end{table}

Since much of the test content had relatively consistent results we did not think more tests was needed. The Instagram images appeared almost instantly ( 1 second) consistently, since we completed the tests manualy, we could not measure finer times, something we thought was not needed due to the nature of the tests. The twitter content were a little slower as expected, but did also appear consistantly at a reasonable time averaging in just under 20 seconds. The soundcloud publishing also happened pretty quick averaging \texttt{6 seconds} As for the stories we published through Digitalt Fortalt, the times were very inconsistent. In three of the tests (\# 7, \#10 and \#12), which was the longest ones, the content had to be uploaded to DF as well as being published, this probably added to the time. All the other tests were performed with the content already uploaded, just to be published, but we still recorded a massive swing in results ranging from \texttt{10 seconds} (\#8) to \texttt{105 seconds} (\#9).


Average: 
\begin{equation}
\frac{133 + 10 + 104 + 135 + 107 + 132}{6} = {103.5}
\end{equation}

With all results clocking in under 150 secons, and our goal being under 300 seconds, we are very pleased with the results and can happily see our app passing this test. Considering the version of the app we recieved sometimes needed multiple days for the results to appea, it is needless to say there have been a massive improvement.


Another important part of the performance efficiency are the application's data usage. Blowing the users' data limit and potentially taking a large part in increasing their phone-bill is something we want to avoid, and to avoid that we have implemented a data usage restraint. This restraint is set through the equation described in \todo{4.12.2}.

We tested the data usage to make sure our application met our standard. We measured this roaming unconnected to any wifi-hotspots while running the app.\\

\emph{Quick session}\\
Opening map: \texttt{55.74 kB}\\
Entering place with only one story: \texttt{+ 6.07 kB}

\emph{Browsing session}\\
Browsing map over Trondheim: \texttt{1.64 MB}\\
Entering place with 6 stories and 2 instagram pictures: \texttt{+ 0.02 MB}

We found these results to be pretty reasonable. The app does not download more content than necassary, and the results seems really consistent compared to the experiences we have had using the application.

\paragraph{Reliability}


\paragraph{Portability}

The multiplatform aspect of the project has played a major part in the development and have played a large part in our choice of environment and frameworks. What we want to achieve is a reasonable consistency through different platforms and versions. We have throughout the development process tested it with many different viritual devices on different settings, but the most valuable ones are the ones performed on the physical devices at our disposal.\\

\begin{table}[!htp]
\begin{center}
	\begin{tabular}{ | l | l | l | r | }
	\hline
	 \#	& Platform (version)	&Device			& Notes \\ \hline
	 1	&Android (4.4.2 KitKat)	&LG Nexus 5			& Works prefectly\\ \hline
	 2	&Android (4.3 Jelly Bean)	&Samsung Galaxy SII	& No notable differences or bugs\\ \hline
	 3&&&\\ \hline
	 4&&&\\ \hline
	 5&&&\\ \hline
	 6&&&\\ \hline
	 7&&&\\ \hline
	 8&&&\\ \hline
	 9&&&\\ \hline
	 10&&&\\ 
	 \hline
	 \end{tabular}
\end{center}
\caption{Portability Testing}
\label{tab:Portability Testing}
\end{table}


\/*

\subsubsection{Acceptance testing}



\begin{center}
\begin{tabular}{ i{4cm} ||  i{10cm}} \toprule
\multicolumn{2}{c}{\textbf{Select a place}} \\ \hline
Related test case (ID) & T-F1 \\ \hline
Test Description & We asked the user to open the app and browse the map to find a desired ``place''. \\ \hline
Steps & \begin{enumerate} \item Open the application \item Wait for response \item Browse the map and find a place of your choosing \item Select place by clicking on pinpoint \end{enumerate} \\ \hline
Result & The user had little issue doing the test, without hesitation or any questions, completeing the task with ease. The response we got was that the interface was very intuitive and that little effort was needed to understand how to proceed.  \\ \hline
Verdict & Passed \\ \bottomrule
\end{tabular}
\end{center}

*/



\printbibheading
\printbibliography[type=book,heading=subbibliography,title={Book Sources}]
\printbibliography[nottype=book,heading=subbibliography,title={Other Sources}]

\clearpage
\section{Documents}

					\begin{center}
						\begin{tabulary}{\textwidth}{L | R | C | C | C | L | R} \toprule
Problem & Description & Likelihood (1-9) & Impact (1-9) & Importance (Likelihood * Impact) & Preventive action & Remedial Action \\ \bottomrule
						\end{tabulary}
						\begin{tabulary}{\textwidth}{L | R | C | C | C | L | R} \\
Communication Loss & Group members doesn't communicate with each other. Group don't establish good communication with the customer and supervisor & 3 & 7 & 21 & Actively establish communication and reach out to the parties. & Talk with the group about the communication, and try to get a good grip of what is failing. Establish communication media, so the group can talk with eachother.\\ 
\hline
Change requests & Change requests that does not meet the requirements of the product & 3 & 7 & 21 & Well defined requirements spesification, implementing it iterative. & Talk with the customer and ask what he thinks about the request changes.\\ 
\hline
Technical difficulties
 & Some problems may turn up to be very hard to solve. This can in turn lead to delays and frustration. And may sometimes be very time consuming. & 5 & 4 & 20 & Regulary have technical discussions with the group, that way the hard problems can be handled by the group as a whole.
 & If the problem is to hard, try to get help from other groups. Also evaluate if the problem can be handled differentely.\\ 
\hline
						\end{tabulary}
						\begin{tabulary}{\textwidth}{L | R | C | C | C | L | R} \toprule
Workstation are noisy & The workstation is filled with people who make alot of sound, so the developers team can't concentrate to the fullest. & 5 & 4 & 20 & Can preorder room, so we get our own workstation to work on. & Preorder room, and move the whole developers team there. If the noise is that bad.\\ 
\hline
Failing to do planned work & Members of the group fails to do schedueld work due to falling behind in subjects not related to the project or other things.  & 9 & 2 & 18 & Good scheduling habits. Sit down every week and see what's planned to do in the project the following week. Coordinate against what you have to do in other subject. & Make up for lost work during weekends or other available time slots\\ 
\hline
Insufficient product & Devolping a product that does not meet the requirements of the costumer & 2 & 9 & 18 & Good communication with the costumer, in sort of agile devolpment such as Scrum & \\ 
\hline
API change & The general API has to be changed, because it lack functionality. & 2 & 9 & 18 & Sufficient research about API before implementing it into the project. & Either drop the functionality that is missing, or start developing with the new API.\\ 
\hline
						\end{tabulary}
						\begin{tabulary}{\textwidth}{L | R | C | C | C | L | R} \toprule
Different app views & Customer and developers have different views of the apps purpose and funtions & 3 & 6 & 18 & Have regular meetings, inform and discuss all changes to project scope, goals and features. & Discuss with customer and find middle ground.\\ 
\hline
Scope & The amount of features requested are beyond what the development team can deliver in time & 6 & 3 & 18 & Be specific with the customer how much time we have, and explain deeply how much time it takes to develop a single feature & Discuss what are the nessasery features that must be in the product, and flush out what is the least nessasery.\\ 
\hline
Lack of competence & Don't have enough competence about the given software we are suppose to use during the project. & 8 & 2 & 16 & Meet every day, do workgroups together and learn by failing. & Talk with other members of the group, and hear if they have the competence. This will prevent hours of searching, when you can listen what the other members have to say. And direct you on the right path for the competence you need.\\ 
\hline
						\end{tabulary}
						\begin{tabulary}{\textwidth}{L | R | C | C | C | L | R} \toprule
Install of stedr & Not all group members can install the app on their own device. The purpose is to evaluate stedr. & 5 & 3 & 15 & Install it while you have a meeting, so all the group members can atleast watch the app on another device. & Go together 2 and 2, and watch the app on a phone whose able to install the app.\\ 
\hline
Missing deadlines & Some work may take longer time than expected, this this may cause delays later on in the project. & 3 & 5 & 15 & Have a steady and diciplined workflow and plan ahead. Overestimate work rather than underestimate. & All members meet and plan what is to be done, and do it at once. So we can deliver as soon as possible.\\ 
\hline
Customer turnover disruption & A key contact in SINTEF leaves the company, putting the project in a unclear state & 2 & 7 & 14 & Good communication. Multiple contacts with knowledge of the project & Quickly contact the customer and discuss how to proceed and how it's affected\\ 
\hline
Sickness & Group members or other crucial personell gets sick & 4 & 3 & 12 & Have regular updates about the progress of the work being done, and don't make important task rely completely on one person without a backup plan. Don't freeze and drink a lot of tea. & Talk to the person about the individual tasks, how much he can handle, and distribute the work the member can't.\\ 
\hline
						\end{tabulary}
						\begin{tabulary}{\textwidth}{L | R | C | C | C | L | R} \toprule
Group members falling out. & Members doesn't show for meetings, or goes of the grid without notice. & 2 & 6 & 12 & Good communication and agree on a schedule that suits everyone. & Take action at once, and make inquires to why the member didn't show.\\ 
\hline
Uneven workload & Uneven distribution of workload & 6 & 2 & 12 & Keep updated on the tasks given and work put in, and distribute work  & Make the member or members direct task. So it's easy for the member or members to do so. \\ 
\hline
Conflict over changes & Group members not in agreement over supposed changes in group management, work, responibility etc. & 3 & 4 & 12 & Have an open dialog. & Discuss in group and decide as a democracy.\\ 
\hline
Late for meeting & Members of the group are late for meetings with group/customer and supervisor & 6 & 2 & 12 & Good communication and agree on a schedule that suits everyone. & Take action at once, and make inquires to why the member came late.\\ 
\hline 
						\end{tabulary}
						\begin{tabulary}{\textwidth}{L | R | C | C | C | L | R} \toprule
Documents customer/supervisor meeting & You lack the sufficeint documents for the meeting with the customer. For presentation on how you want the app to be, mockups and reports about fieldwork etc. & 2 & 6 & 12 & Have the documents stored in the cloud so  you can acces it where ever you go. With your respective smartphone/tablet and pc's, & Discuss what you remember and try to make the best out of the meeting, as possible.\\ 
\hline
Equipment failure & Computers and other dependable devices malfunctions. & 4 & 2 & 12 & Keep documents and code in the cloud so you can work from another device if your primary device malfunction. & Get replacement as soon as possible.\\ 
\hline
Document sharing failed & Authorization of documents sharing is not complete, people don't have access to the groups documents. & 2 & 4 & 8 & Give all the authorization they need for the documents to be shared. So all can view, edit and share documents. & Find out where the problem lies, so everyone can get authorization for the given documents and folders.\\ 
\hline
						\end{tabulary}
						\begin{tabulary}{\textwidth}{L | R | C | C | C | L | R} \toprule
Lack of software  & nessasery for the develoment progress & 1 & 3 & 3 & Talk about what software is required for the development of the product. Ask the customer for this software.   & Ask the customer immediately for the required software, so the development progress don't have any major delays. \\ 
\hline

						\end{tabulary}
					\end{center}

\section{Attachments}

\begin{figure}[!h]
\centering
\includegraphics[width=1.3\textwidth, angle=270] {class-frontend.png}
\caption{Class diagram for frontend}
\end{figure}

\begin{figure}[!h]
\centering
\includegraphics[width=1.3\textwidth, angle=270] {class-backend.jpg}
\caption{Class diagram for frontend}
\end{figure}

\appendix

\section{test}

\end{document}
\section{Architecture}
\thispagestyle{plain}

The current architecture of the application is as shown in figure 3 \ref{fig:fig1}. We have a back-end written in Java that retrieves information from services like Digitalt Fortalt, Flickr and Instagram. Digitalt Fortalt is where all the stories are obtained from. Flicker holds all the locations, and the pictures are taken from Instagram based on tags. The information is stored on the server and can now be used by the client, which holds the front-end of the application that is being developed on Appcelerator Titanium, using mainly JavaScript and XML. Twitter is integrated directly into the front-end and does not have to go through the server. In addition, we have also added support for Soundcloud. 

\vspace{0.4in}
\begin{figure}[!h]
\begin{center}
\includegraphics[width=1\textwidth]{res/DeploymentView.jpg}
\caption{Deployment View}
\end{center}
\end{figure}


\subsection{Back-end}
The Back-end is written in Java and mainly retrieves data from external APIs and save it on the server so that it can be used by the application.

\begin{figure}[!h]
\begin{center}
\includegraphics[scale=0.15]{classBack.jpg}
\caption[miniature class diagram: back-end]{Full scale class diagram can be found in Attatchments \ref{sec:classDiagrams}}
\end{center}
\end{figure}

\subsection{Front-end}
The front-end of the application is an interface to let the user enter, manipulate and view data. It is the part of the application that is being interpreted on the users own device, and is based on XML, TSS and JavaScript for design and functionality. 

Every window in the application has a JavaScript-, TSS- and XMLl-file associated with it. A window can contain various views that can each have different event listeners. What the user sees depends on the window currently open and its associated XML, TSS and JavaScript files. What happens when interacting with a view depends on the event-listeners attached to that particular view. Interactions can be purely visual or it can trigger core functionalities. For example the refresh button on the map window has an event-listener attached to it so that when the user clicks it, it will attempt to fetch the locations from the server and plot them on the map. It will also animate the refresh icon to spin, giving the user feedback when the click was registered.


\begin{figure}[!h]
\begin{center}
\begin{tabular}{cc}
\includegraphics[scale=0.15]{classFront1.png}&
\includegraphics[scale=0.15]{classFront2.png}\\
\end{tabular}
\caption[miniature class diagram: front-end]{Full scale class diagram can be found in Attachments \ref{sec:classDiagrams}}
\end{center}
\end{figure}

\subsection{Process View}

\begin{figure}[!h]
\begin{center}
\includegraphics[width=1\textwidth]{res/processView.pdf}
\caption{Process View}
\end{center}
\end{figure}


\subsection{Use Case}

\begin{figure}[!h]
\begin{center}
\includegraphics[scale=0.6]{ms.png}
\caption{Map View (Home)}
\end{center}
\end{figure}

\begin{figure}[!h]
\begin{center}
\includegraphics[scale=0.6]{mens.png}
\caption{Menu View}
\end{center}
\end{figure}

\begin{figure}[!h]
\begin{center}
\includegraphics[scale=0.6]{ps2.png}
\caption{Place Screen}
\end{center}
\end{figure}

\clearpage

\subsection{Sequence}

\begin{figure}[!h]
\begin{center}
\includegraphics[scale=1]{Get-Stories}
\caption{Get Stories}
\end{center}
\end{figure}

\begin{figure}[!h]
\begin{center}
\includegraphics[scale=1]{Get-Collections}
\caption{Get Collections}
\end{center}
\end{figure}
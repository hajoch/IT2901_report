\section{Evaluation}
\thispagestyle{plain}
	\subsection{Process}

With our process model, Scrum, we found that following it by the book, became very troublesome. Therefore we decided to make some modification to the original model. The main problem with using Scrum by the letter is that we are all students, and this project only counts for half of the semesters study points. This means we all have different schedules, and thus making it difficult to have daily meetings. End meetings, or retrospective meetings is also something we have not prioritized much.

In Scrum it is common to use something called planning poker when deciding how long tasks should take. This essentially means that everyone “votes” on how time consuming they think a given task will be. We found this to be a little unnecessary because its usually so imprecise, and have therefore chosen to just let the persons responsible give their judgements to save time. 

	\subsection{Project Management}

One of the main challenges for us concerning project management was the changes made in regards to features during the first part of the project. At the beginning our main customer contact was sick, and the replacement contact was much less acquainted with the project, and from these talks we got the impression that it was wished that we had our own proposals of features, and we were encouraged to start working in the direction we thought most likely to benefit the vision the customer had for the app. When our main customer contact became well again we understood that her vision on the app differed from the impression we previously had gotten in some significant respects, and that she had a clearer idea about what features should be included in the app than our previous customer contact. We take self criticism for not being more careful in the beginning in regards to starting to work on features that we weren’t sure to be prioritized by the customer. It also took a while to come to a common understanding with our customer of the requirements. A while into the project, in order to help us make up for a slow start, the customer wrote a detailed, unambiguous and prioritized list of requirements, including about 50 features. This was a breakthrough in our communication with the customer that helped us reach a common understanding of both the large picture and the details.

	\subsection{Communication}

We feel that the communication internally on the team have been good. By using different channels like Facebook, mail, and phone, we were able to stay in touch throughout the project even though we couldn't always be physically present on the school.

With the customer much of the communication has been good and effective, but parts of it could have been handled better if we were to start over again, and was obviously halted in the beginning by our main customer contact being sick. Due to this sick leave, our communication had to go through a third party. At first, it didn't seem like a big problem, but when we later discovered how this had led to major misunderstandings, it ended up hurting us more than we first thought it would.

About three weeks into the project, we started to have direct communication with our customer over Skype once a week. These meetings didn't work well for us. It made it hard for everybody to engage in the conversations. Although we always paid attention and were careful to write summaries from these meetings, the customer's visions didn't get trough to us somehow. We thought we had a clear picture of what our customer wanted, but it turned out to be wrong.

It was not really until we started meeting our customer in person, on March the 5th, that we understood how much we had been talking past each other. There had been lacking clarity in messages between us and since both parts thought they understood each other, no measures where made. Now, after an almost four hour meeting with our customer in person, we finally came to an common understanding of what purpose the app had. This also meant that we suddenly was far behind schedule since we had to completely redo the product backlog based on the new feature list. Had such a list been provided to us from the beginning, much time would have been saved.

That being said, we also take self criticism. There was for instance one incident where a team members didn't pay enough attention on a meeting. This resulted in him working almost 20 hours with trying to integrate the SoundCloud UI experience into Stedr. Our customer had previous stated that it was enough for us to just make a simple API call to the SoundCloud servers and retrieve sounds based on title. Additionally, our customer felt that we sometimes were slow to answer emails. This is regrettable, and we tried to improve in this regard after she mentioned this.

	\subsection{Project planning}

Our pre-study and project planning was really thorough, but in retrospective, much of it ended up being a waste. We spent a lot of time doing research and user tests, trying to figure out what direction we should take the app in. Because of the misunderstanding explained in Communication, we thought we had much more freedom than we actually had. We had the idea that we could just get creative and play with the app as we saw fit. We do not have any problems working with predefined feature lists and requirements, but then it should be clear from the start that we in fact have these restraints.

In the first half of the project, these were some of the features we were working on that we later dismissed because 1) they didn't fit into what the app is about, and 2) we misunderstood some of the requirements.

\begin{itemize}
	\item Adding new places from the app
	\item Wikipedia integration - Possibility to see wiki entries related to the place you are visiting.
	\item Adding full SoundCloud experience in the app.
	\item Attaching NRK archive footage to a place. 
	\item General design overhaul.  (Dismissed mock-ups can be found in \ref{sec:mockups})
\end{itemize}

	\subsection{Problems and difficulty}

The development of this app have been a bit bumpy for us. We have stumbled upon surprisingly many problems from our risk analysis. One could of course question our preventive abilities, but we still feel we took the right precautions. Some things are just left to luck.

From out risk analysis, this is some of the more important problems we came across in this project. 

\begin{itemize}
	\item Communication failure - Between the team and the customer as explained in Communication.
	\item Major requirements change - For us, the new feature list meant we had to change the priorities of the requirements a lot.
	\item Technical difficulties - We had huge trouble setting up the development environment for titanium framework. Two team members didn't even get it to work at all.
	\item Unavailability - Two team members spent 2 weeks in China which reduced our capacity right before Easter. 
	\item Lack of Competence - Combined we had zero experience with some of the technologies used in this project beforehand.
	\item Sickness - As explained above.
	\item Equipment Failure - One team member had to send his computer to service for a total of 6 weeks. Two others had to replace their android device. 
\end{itemize}

This was also the first time for all of us to take over a project halfways to further develop it. This meant that we didn't have the option to choose technologies based on our strengths since its already chosen by the previous group. Actually, none in our group had any experience with any of the technologies used. This was very unfortunate since we have to spend a lot of time to learn new frameworks. Additionally, you have the aspect of understanding all the code that had been written. 

	\subsection{Lesson learned}

In terms of management, we have learned some important lessons.

Firstly is the importance of clear milestones. In the beginning we had very unclear goals and this affected the group. It was later solved when we got a prioritized list \ref{sec:featList} from our customer. Working without a clear focus can be challenging for the team members.

Also, clarity internally. If we had been a bit more clear on responsibilities within the group, the development process would have gone more smoothly.

Lastly, we cannot overemphasize this: Communication is a determining factor for a projects success that we should not underestimate the importance of. Of course we already knew this coming into the project, but in practice it is easy to lose focus. Because of a simple misunderstandings we spent two weeks working on the user interface and other irrelevant features when we should have prioritised integrating APIs instead. So clarity on what the customer want is important. We also learned that Skype meetings or talking over the phone is not sufficient for good communication.

On the technical side, when it came to development platforms, we initially didn't have much to say, since the previous group had already made the decision to use Titanium front-end and Play back-end.

In the aftermath of the project, our general consensus is that Titanium is a sub-optimal framework for this kind of project. Stedr is an app that uses many different APIs and integrating these in a good way turned out to be hard. This is because Titanium provides a kind of half-breed between native and web development. Most APIs are made for either native or web and does not translate well into the Titanium environment, thus making it difficult to use. Just setting up the work environment turned out to be quite a task, especially on some of our team members machines. 

It would, in our opinion, been better to work native or just use pure web-standard (HTML5, CSS3 and JavaScript).

	\subsection{Technical evaluation and recommendations}

After reviewing the test cases the system seems to satisfy the requirements (and features) set by the customer, except for the non-functional requirement related to reliability.

The failing of the reliability is caused by the server, because of the lack of storage and the servers dependency of external systems. Especially the external system used for storage of the Places (Flickr) has been a problem, this is because the external system uses very long time to respond to our internal system's requests. This problem will also probably magnify in size as more Places are added to Flickr, which in turn will lead to an even greater delay of the response sent by the external system.

To enhance the systems reliability the customer should evaluate the possibility to add a database back-end. This would make it possible for the system to store it's own information so that the only dependency and single point of failure would have been the server itself. A requirement for this, is that the customer has to take editorial responsibility for the content stored on the server which the customer at the time of this project wasn't willing to do.

Another recommendation for further work is that the option to remove the back-end as a whole. As of now the system roughly is a Model View Controller-application, where the smartphone application acts as the view. The computational restrictions is therefore back-end, but the back-end is limited to run as a cloud service with limited resources as it should be free of charges. A consequence is that it becomes natural to ask if the processing done back-end, couldn't be done more efficient at the end-users smartphone. Combining the removal of a back-end with local storage, would also lead to the application no longer require continuous Internet access.

Both of these recommendations implicates that some parts of, or the whole system, needs to be rewritten. Careful evaluation of the options needs to be done before eventually deciding if the reliability of the system isn't good enough.

\subsection{Conclusion}

Even though we were both unfortunate and a little careless in the beginning, we managed to pull ourself together and produce a result we are really proud of. All that is left now, is to hope our customer feels the same about our work, and like us feels that we had a working relationship that on balance was pleasant, fruitful and successful. We have learned incredibly much from this project since we have stumbled across a fair amount of unlikely, but yet realistic problems that can occur in every working project. We feel that this valuable experience will help us to improve greatly for future projects.
